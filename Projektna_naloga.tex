\documentclass{article}
\usepackage[slovene]{babel}
\usepackage[a4paper,top=2cm,bottom=2cm,left=3cm,right=3cm,marginparwidth=1.75cm]{geometry}
\usepackage{amsmath, amsthm, amsfonts, amssymb}
\usepackage{float}
\usepackage{graphicx}
\usepackage{mathtools}
\usepackage{hyperref}

\DeclareMathOperator{\se}{SE}
\DeclareMathOperator{\de}{p}
\DeclareMathOperator{\CM}{C}
\DeclareMathOperator{\pval}{p}

\title{Projetkna naloga pri statistiki}
\author{Gaja Jamnik}
\date{}

\begin{document}
    \maketitle

%%%%%%%%%%%%%%%%%%%%%%%%%%%%%%%%%%%%%%%%%%%%%%%%%%%%%%%%%%%%%%%%%%%%%%%%%%%%%%%%%%%%%%%%%%%%%%%%%%%%%%%%%%

\section{Prva naloga}

Pri prvi nalogi smo obravnavali podatke 43.886 družin iz mesta Kibergrad. Natančneje, analizirali smo dohodke
družin in njihovo porazdelitev primerjali s ustrezno normalno porazdelitvijo.

\subsection{A del naloge}

Narisali smo histogram dohodkov vseh družin v Kibergradu. 
Širino in število razredov v histogramu smo določili s pomočjo formule
\begin{equation}
    l = \frac{2(q_{3/4} - q_{1/4})}{\sqrt[3]{n}},
\end{equation}

kjer sta $q_{\frac{1}{4}}$ in $q_{\frac{3}{4}}$ prvi in tretji kvartil, n pa je število enot.
Tej formuli pravimo Freedman–Diaconisovo pravilo. Po formuli znaša širina posameznega razreda približno $2128$.
Glede na interval na katerem so razporejeni dohodki znaša to ravno $223$ razredov v histogramu.

\begin{figure}[H]
    \begin{center}
        \includegraphics*[width=0.5\textwidth]{figure1A.png}
        \caption{Histogram dohodkov družin v Kibergradu}
        \label{hist_dohodki}
    \end{center}
\end{figure}

\subsection{B del naloge}

V B delu naloge smo v histogram dorisali še normalno gostoto, katere pričakovana vrednost se ujema s povprečjem
dohodkov družin v Kibergradu, standardni odklon pa s standardnim odklonom podatkov o dohodkih družin.
Izračunano povprečje podatkov je $41335.50704096979$, standardni odklon pa $32037.61941788666$.

\begin{figure}[H]
    \begin{center}
        \includegraphics*[width=0.5\textwidth]{figure1B.png}
        \caption{Histogram dohodkov družin v Kibergradu in normalna gostota}
        \label{hist_in_normalna}
    \end{center}
\end{figure}

Histogram je desno asimetričen. V primerjavi z grafom normalne gostote je zamaknjen v desno, 
po obliki se desni del histograma lepo prilega krivulji.

\subsection{C del naloge}
V tem delu naloge primerjamo kumulativni porazdelitveni funkciji podatkov o dohodkih in ustrezne normalne
porazdelitve, ki ima za pričakovano vrednost povprečje dohodkov družin v Kibergradu in standardni odklon, ki se ujema
s standardnim odklonom dohodkov. Kumulativno porazdelitveno funkcijo porazdelitve dohodkov definiramo s pomočjo
funkcije arange, tako, da definiramo seznam deležev. Za normalno porazdelitev pa uporabimo kar že vgrajeno funkcijo
statistics.NormalDist(mean, std).cdf(x).
\begin{figure}[H]
    \begin{center}
        \includegraphics*[width=0.5\textwidth]{figure1C.png}
        \caption{Primerjava kumulativnih porazdelitvenih funkcij}
        \label{kumulativne}
    \end{center}
\end{figure}

Grafa se od vrednosti $150000$ skoraj popolnoma prilegata. Na levi strani pa je nekaj manjših odstopanj.


\subsection{D del naloge}
S primerjalnim kvantilnim grafikonom oz. Q-Q grafikonom smo primerjali kvantile normalne porazdelitve s kvantili 
porazdelitve dohodkov družin v Kibergradu.
\begin{figure}[H]
    \begin{center}
        \includegraphics*[width=0.5\textwidth]{figure1D.png}
        \caption{Primerjalni kvantilni grafikon}
        \label{QQ}
    \end{center}
\end{figure}

Če bi bili dohodki normalno porazdeljeni, bi kvantilni grafikon, narisan z modro, sovpadal s 45-stopinjsko premico,
narisano z rdečo barvo. Iz grafikona razberemo, da dohodki niso porazdeljeni normalno.

\subsection{E del naloge}
Nato smo analizirali porazdelitev dohodkov z enostavnim slučajnim vzorčenjem. Izbrali smo $1000$ enostavnih
slučajnih vzorcev velikosti $400$. To smo naredili s pomočjo vgrajene funckije random.randint. 
Nato smo narisali histogram vzorčnih povprečij dohodkov.
\begin{figure}[H]
    \begin{center}
        \includegraphics*[width=0.5\textwidth]{figure1E.png}
        \caption{Histogram vzorčnih povprečij}
        \label{hist_vzor}
    \end{center}
\end{figure}

\subsection{F del naloge}
Na histogram smo dorisali graf normalne gostote. Za pričakovano vrednost smo vzeli povprečje dohodkov
celotne populacije, za standardni odklon pa standardno napako nekega slučajnega vzorca velikosti $400$.
Vzorec smo ponovno generirali s pomočjo funkcije random.randint. Standardno napako, pa smo glede na vzorec 
izračunali s funkcijo sem(). 

\begin{figure}[H]
    \begin{center}
        \includegraphics*[width=0.5\textwidth]{figure1F.png}
        \caption{Histogram vzorčnih povprečij z normalno gostoto}
        \label{hist_vzor_normalna}
    \end{center}
\end{figure}

Graf gostote normalne porazdelitve je narisan z rdečo barvo. Graf se histogramu ne prilega..
Za primerjavo sem še posebej narisala graf dane normalne porazdelitve. 

\begin{figure}[H]
    \begin{center}
        \includegraphics*[width=0.5\textwidth]{figure1F(1).png}
        \caption{Gostota normalne porazdelitve}
        \label{normalna}
    \end{center}
\end{figure}
Opazimo lahko, da imata graf in histogram vrh pri približno enaki vrednosti abscise, vendar je vrh normalne 
gostote znatno nižji.

\subsection{G del naloge}
Tudi v tem delu naloge smo uporabili normalno porazdelitev s pričakovano vrednostjo,
ki se ujema populacijskem povprecju dohodkov in standardnim odklonom, ki se ujema s standardno napako za 
enostavni slučajni vzorec velikosti $400$.
Za vzorčna povprečja smo narisali kumulativno porazdelitveno funkcijo. V isti graf smo za primerjavo z 
rdečo barvo narisali še kumulativno porazdalitveno funkcijo zgoraj opisane normalne porazdelitve.
Iz grafa \ref*{CDFpovprecja} opazimo, da se funkciji dobro prilegata.
\begin{figure}[H]
    \begin{center}
        \includegraphics*[width=0.5\textwidth]{figure1G.png}
        \caption{Kumulativna porazdelitvena funkcija za vzorčna povprečja (z modro) in kumulativna porazdelitvena funkcija normalne porazdelitve (z rdečo)}
        \label{CDFpovprecja}
    \end{center}
\end{figure}

Narisali smo še primerjalni kvantilni grafikon s katerim smo primerjali porazdelitev vzorčnih povprečij z
normalno porazdelitvijo.
\begin{figure}[H]
    \begin{center}
        \includegraphics*[width=0.5\textwidth]{figure1G(1).png}
        \caption{Kvantilni grafikon vzorčnega povprečja v primerjavi z normalno porazdelitvijo }
        \label{QQvzor}
    \end{center}
\end{figure}
Na grafu \ref*{QQvzor} se kvantilni grafikon, narisan z modro, dobro prilega rdeči 45-stopinjski premici.
To pomeni, da ima vzorčno povprečje normalno porazdelitev.

\section{Naloga 2}

V drugi nalogi smo analizirali podatke o deležu mangana v železu, pridobljenem v plavžu.
Podatki so vsebovali deleže mangana v odlitkih, ki so jih jemali skozi 24 dni, petkrat dnevno.
Preučiti smo želeli normalnost dobljene empirične porazdelitve. Porazdelitev deležev
mangana smo primerjali z normalno porazdelitvijo, katere pričakovana vrednost in standardni odklon
se ujemata s povprečjem in standardnim odklonom deležev mangana iz raziskave.

Najprej smo narisali histogram vseh deležev z dorisano normalno gostoto. Širino posameznega razreda v histogramu 
smo kot v nalogi 1 izračunali s pomočjo Freedman-Diaconisovega pravila. Širina razreda je znašala
$0.07298642394688079$, kar je ravno $11$ razredov.

\begin{figure}[H]
    \begin{center}
        \includegraphics*[width=0.5\textwidth]{figure2A.png}
        \caption{Histogram deležev mangana v plavžu z dorisano normalno gostoto (z rdečo)}
        \label{2A}
    \end{center}
\end{figure}

Graf normalne gostote se lepo prilega histogramu. Vrh histograma in grafa sta dosežena pri 
približno enaki vrednosti na abscisi. Histogram je tudi zvončaste oblike, kar namiguje na normalno porazdelitev.

Vendar to opažanje ni dovolj, saj normalna porazdelitev ni edina z zvončasto obliko histograma.
Zato smo za primerjavo uporabili še viseči histogram razlik korenov frekvenc.
Klasičen histogram ponazori frekvence posameznih vrednosti iz podatkov. 
Viseči histogram pa primerja korene frekvenc dane porazdelitve s frekvencami ustrezne normalne porazdelitve.
To naredimo tako, da najprej poračunamo verjetnosti, da ima normalno porazdeljena spremenljivka 
vrednost na danih inetrvalih. Ocena verjetnosti na intervalu $[x_{j-1}, x_j]$ bo
\begin{equation}
    \hat{p_j} = \Phi \biggl(\frac{x_j - \overline{x}}{\hat{\sigma}}\biggr) - \Phi\biggl(\frac{x_{j-1} - \overline{x}}{\hat{\sigma}}\biggr),
\end{equation}
kjer je $\overline{x}$ povprečje naših podatkov, $\hat{\sigma}$ pa ocena za standardni odklon.

Nato določimo predvideno število pojavitev na $j$-tem intervalu kot
\begin{equation}
    \hat{n_j} = n\hat{p_j},
\end{equation}
kjer je $n$ število podatkov o deležih mangana.
V zadnjem koraku poračunamo razliko korena števila ponovitev vrednosti na $j$-tem interval,
kar označimo z $n_j$ s korenom
ocenjenih ponovitev normalne porazdelitve:
\begin{equation}
    \sqrt{n_j} - \sqrt{\hat{n_j}}.
\end{equation}
Poračunane vrednosti narišemo v histogram.
Iz grafa \ref{2B} vidimo, da imamo vzdolž vseh vrednosti velika odstopanja v frekvencah.
\begin{figure}[H]
    \begin{center}
        \includegraphics*[width=0.5\textwidth]{figure2B.png}
        \caption{Viseči histogram razlik korenov frekvenc}
        \label{2B}
    \end{center}
\end{figure}



Za nadaljno primerjavo uporabimo še kvantilni (Q-Q) grafikon. 
\begin{figure}[H]
    \begin{center}
        \includegraphics*[width=0.5\textwidth]{figure2C.png}
        \caption{Primerjalni kvantilni grafikon}
        \label{2C}
    \end{center}
\end{figure}

Iz grafikona \ref{2C} opazimo, da se kvantili prilegajo $45$-stopinjski premici, na robovih pa pride do 
rahlih odstopanj. Kvantilni grafikon kaže, da so deleži mangana porazdeljeni približno normalno.

%%%%%%%%%%%%%%%%%%%%%%%%%%%%%%%%%%%%%%%%%%%%%%%%%%%%%%%%%%%%%%%%%%%%%%%%%%%%%%%%%%%%%%%%%%
\section{Naloga 3}
V tretji nalogi smo analizirali
podatke o dolžini zob morskih prašičkov, ki so jim
dodajali vitamin C v različnih količinah na dva različna načina: bodisi neposredno
bodisi s pomarančnim sokom.

\subsection{A del naloge}
Najprej nas je zanimalo ali dodajanje vitamina C sploh vpliva na rast zob. Za začetek smo izračunali
povprečno dolžino zobu, glede na količino dodatka vitamina C. Za količino $0.5$ je povprečna dolžina zoba
$10.605$, za količino $1.0$ je $19.735$, za količino $2.0$ pa $26.100$. To nam da misliti, da dolžina zobu raste 
z količino dodatka vitamina C. Preizkusimo to še s preizkusom hipoteze. Definirajmo ničelno in alternativno
hipotezo:
\begin{align}
    H0: \ \mu_{0.5} \geq \mu_{1.0} \\
    H1: \ \mu_{0.5} < \mu_{1.0}.
\end{align}
Hipotezo preizkusimo s Studentovim t-testom. 
Izberimo stopnjo tveganja $\alpha=0.05$.
Naj slučajna spremenjlivka $X$ označuje porazdelitev velikosti zob pri dodajanju količine $0.5$ vitamina C,
spremenljivka $Y$ pa pri dodajanju količine $1.0$. 

Na vajah smo pokazali, da za spremenljivko T velja
    \[
        T \coloneqq \frac{\overline{X} - \overline{Y}}{S}\sqrt{\frac{mn}{m + n}} \sim Student(n + m - 2),
    \]
    kjer je
    \[
        S \coloneqq \sqrt{\frac{\sum_{i = 1}^n (X_{i} - \overline{X})^2 + \sum_{i = 1}^m (Y_{i} - \overline{Y})^2}{m + n - 2}}.
    \]
V našem primeru je $n=m=30$.
Iz formule
\[
        P(|\overline{M} - \overline{Z}| \leq \CM_{\alpha}) \leq \alpha 
    \]
nato poračunamo $C_{\alpha}$. Iz vzorcev izračunamo povprečji $\overline{X}$ in $\overline{Y}$, ter nato 
na podlagi teh podatkov in $C_{\alpha}$ izračunamo zgornjo vrejetnost $p$. Če je $p < \alpha$ ovržemo
ničelno hipotezo, sicer jo obdržimo. 
Verjetnost smo izračunali kar z že vgrajeno funkcijo $stats.ttest_ind()$. 
Dobimo vrednost $p = 6.331484806608257e-08$,
kar je manjše kot $\alpha$, zato zavržemo ničelno hipotezo. 

Podobno storimo še s primerjavo pričakovane vrednosti dolžin zob z dodajanjem količine $1.0$ in $2.0$.
Z enakim postopkom dobimo $p=9.054142680908653e-06$, kar je manj kot $\alpha$, zato bo veljalo, da je 
$\mu_{1.0} < \mu{2.0}$. Od tod sledi, da res dodajanje vitamina C pozitivno vpliva na rast zob.


\subsection{B del naloge}
Dalje nas je zanimalo, kateri način dodajanja vitamina C je učinkovitejši. Ponovno naredimo preizkus hipoteze s
Studentovim t-testom.
Naj $P$ označuje slučajno spremenljivko za dolžino zob pri dodajanju s pomarančnim sokom, $N$ pa pri neposrednem dodajanju.
Zapišimo hipotezi:
\begin{align}
    H0: \mu_P \le \mu_N \\
    H1: \mu_P > \mu_N
\end{align}

Ponovno uporabimo vgrajeno funkcijo in dobimo $p=0.03019668561206424$, kar je manjše kot $\alpha=0.05$.
Torej ničelno hipotezo ponovno zavržemo. Sledi, dodajanje s pomarančnim sokom je učinkovitejše.
Vendar bi za stopnjo tveganja $\alpha=0.01$ ničelno hipotezo držali, in trdili bi nasportno.





\end{document}